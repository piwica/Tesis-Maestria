%\begin{center}
%\large \bf \runtitulo
%\end{center}
%\vspace{1cm}
\chapter*{\runtitulo}

\noindent La lista de problemas es el componente estructural de la historia clínica electrónica del Hospital Italiano de Buenos Aires, en ella se detallan los hallazgos y observaciones de los pacientes. Se presenta un análisis basado en teoría de grafos con el objetivo final de encontrar agrupaciones significativas entre problemas antes del 2016. En este modelo los problemas son los nodos y los enlaces son los vínculos con SNOMED CT y la co-ocurrencia en los pacientes. Este análisis comprende la construcción de subconjuntos de los contextos: servicios de atención de salud, nivel asistencial o ámbito y grupo etario. Para evaluar la capacidad predictiva de las agrupaciones se utiliza las métricas de precisión y exactitud en una lista de 10 predicciones seleccionadas en la lista de problemas de paciente en el año 2017. Los resultados mostraron que realizar la lista de predicciones usando sólo los problemas de los contextos, mejora significativamente la capacidad predictiva de las agrupaciones, especialmente en el contexto de servicio de atención de salud y grupo etario.


\bigskip

\noindent\textbf{Palabras claves:} Graph mining, lista de problemas, redes complejas, detección de comunidades, SNOMED CT.