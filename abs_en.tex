%\begin{center}
%\large \bf \runtitle
%\end{center}
%\vspace{1cm}
\chapter*{\runtitle}

\noindent The problem list is the structural component of the electronic medical record of the Hospital Italiano de Buenos Aires, in which the clinical findings and observations of the patients are detailed. An analysis based on graph theory is presented, with the final purpose of finding significant clusters of problems before 2016. In this model, the problems are the nodes, and the links are the connections with SNOMED CT and the co-occurrence in the patients. This analysis includes the construction of subsets with contexts: health care services, level of care or scope and age group. To evaluate the predictive capacity of the clusters, precision and accuracy metrics are used in a list of 10 predictions selected in the problem list of patients in the year 2017. The results showed that making the list of predictions using only the problems of the contexts, significantly improves the predictive capacity of the clusters, especially in the context of health care service and age group.

\bigskip

\noindent\textbf{Keywords:} Graph mining, Problem list, Complex networks, Community detection, SNOMED CT.